\documentclass[letterpaper,11pt]{texMemo}
\usepackage{graphicx}
\usepackage[hyphens]{url}
\usepackage{hyperref}
\usepackage{xcolor}
  \hypersetup{backref=true,
         pagebackref=true,
         hyperindex=true,
         colorlinks=true,
         breaklinks=true,
         urlcolor= blue,
         linkcolor= blue,
         bookmarks=true,
         bookmarksopen=false,
         citecolor=black,
         linkcolor=black,
         filecolor=black,
         citecolor=blue,
         linkbordercolor=blue
}
\usepackage{tabularx}


\memoto{P. Ellison, Executive Director, Hand2Paw Foundation}
\memofrom{sb swae, Researcher}
\memosubject{First Sale Doctrine and the Fair Use Defense}
\memodate{06 May 2018}
\logo{\includegraphics[width=0.25\textwidth]{hand2paw.png}}

\begin{document}
\maketitle
\bibliographystyle{unsrt}

  Reference is to a hypothetical situation in which a 501(3)c not-for-profit corporation seeks information as to whether or not it may procure MLB, NFL, or other major sporting organization-branded material and make, manufacture, distribute, or sell derivative works or goods using the licensed, copyrighted, or trademarked material as raw stock for its goods. This memorandum seeks to provide information, for educational purposes only, regarding the legal issues pertaining to such activities.

  A detailed survey has not been completed at this time. The information presented in this memorandum constitutes the product of research using publicly-available information obtained so far from knowledgeable and responsible sources and makes this summary reply possible.

  The following conclusions can be reported:

  \begin{itemize}
   \item A United States Supreme Court decision in 2013 reaffirmed the first sale doctrine, under which it is generally possible to purchase copyrighted or trademarked material and resell it to other persons. This case is discussed in more detail below.
   \item There exist licensure and/or permission requirements for certain parties wishing to purchase and resell copyrighted or trademarked material.
   \item Broadly speaking, it is not legal for a person to manufacture or otherwise make and subsequently sell derivative works using copyrighted or trademarked material without the permission of the rights holder. Exceptions do exist.
   \item Procuring licensed, copyrighted, or trademarked raw material and making, manufacturing, distributing, or selling derivative works made from the source material will almost certainly rely on a``fair use'' defense in order to assert legality.
  \end{itemize}

  As for specific questions surrounding this matter, the following brief answers develop from the studies made during the past few days. These conclusions are subject to expansion and more detailed examination should the survey be authorized to continue.

\section*{Overview of Copyright}
  Copyrights are a basket of rights that create a legal monopoly on the rights to sell, reproduce, and distribute a given work product. If one wishes to sell goods or material in which the copyright is held by another person or to sell a \emph{de novo} work product that incorporates or derives from such goods or material, two approaches are available. One may make use of exceptions in the copyright laws or one may obtain authorization from the holder of the copyrights.
\section*{First Sale Doctrine}
  The first sale doctrine, which is codified at 17 USC \textsection∗ 109,
     \footnote{17 US Code \textsection∗ 109 - Limitations on exclusive rights: Effect of transfer of particular copy or phonorecord Retrieved from \href{https://www.law.cornell.edu/uscode/text/17/109}{https://www.law.cornell.edu/uscode/text/17/109}.
}
 is a statutory restriction that Congress has placed on the rights of copyright owners. It says that a person who purchases a piece or copy of copyrighted material or goods from the legal holder of the copyright then obtains the right to sell, display, or otherwise dispose of that particular copy.
     \footnote{United States Department of Justice, \emph{Criminal Resource Manual} \textsection∗ 1854 Copyright Infringement -- First Sale Doctrine. Retrieved from {\href{https://www.justice.gov/usam/criminal-resource-manual-1854-copyright-infringement-first-sale-doctrine}{https://www.justice.gov/usam/criminal-resource-manual-1854-copyright-infringement-first-sale-doctrine}}.
}
  Distribution rights end, however, once the owner (purchaser) sells that particular copy. The first sale doctrine does not offer protections for purchasers who make unauthorized reproductions of a copyrighted work.


  The matter of first sale doctrine creates much confusion among copyright holders, purchasers, and prosecutors. In cases that do not involve reproduction of copyrighted materials, it is possible for purchasers/resellers to resist prosecution by making the claim that they reasonably believed that the goods they were selling had previously been subject to a legitimate first sale. Additionally, several criminal convictions related to copyrighted materials have been overturned owing to deficits in the prosecution's proof of the first sale.

  The Owner's Right Initiative, an organization composed of individuals, businesses, and associations that seeks to protect ownership rights in the United States, distill the first sale doctrine down to their slogan, ``You bought. You own it. You have a right to resell it.''
     \footnote{Owner's Rights Initiative. \emph{Our Issues.} Retrieved from \href{http://www.ownersrightsinitiative.org/background/}{http://www.ownersrightsinitiative.org/background/}.
      }
    \footnote{Anita Campbell. \emph{Supreme Court: You Bought It, You Own It, You Can Resell It.} Retrieved from \href{https://smallbiztrends.com/2013/03/resale-rights-you-bought-own.html}{https://smallbiztrends.com/2013/03/resale-rights-you-bought-own.html}.
      }
  It is on this basis that individuals may purchase MLB or NFL branded items and resell them on eBay.

  \subsection*{First Sale Doctrine in the Kirtsaeng v. John Wiley \& Sons Supreme Court Case}
    The United States Supreme Court decided a case in 2013
    \footnote{The Supreme Court of the United States. \emph{Kirtsaeng v. John Wiley \& Sons, Inc., 568 U.S. 519} (2013). Retrieved from \href{https://www.supremecourt.gov/opinions/12pdf/11-697_4g15.pd}{https://www.supremecourt.gov/opinions/12pdf/11-697_4g15.pd}.
    }
    \footnote{Justia. \emph{Kirtsaeng v. John Wiley \& Sons, Inc., 568 U.S. \_\_\_ (2013)} Retrieved from \href{https://supreme.justia.com/cases/federal/us/568/11-697/}{https://supreme.justia.com/cases/federal/us/568/11-697/}.
    }
    that reaffirmed the doctrine of first sale.

    A Thai graduate student was accused of infringing the copyrights of academic publisher John Wiley \& Sons because he made legitimate purchases of Wiley textbooks in Thailand, where they were cheaper, imported them into the United States, and sold them online to US buyers. Wiley claimed that the sales were unauthorized. Kirtsaeng claimed that he had the right to sell the books because they were lawfully purchased in the first place.

    Lower courts ruled against Kirtsaeng on the basis that, because the books were manufactured in a foreign country, he had violated the copyright laws in the United States by conducting unauthorized sales. The Supreme Court held that the relevant language in 17 USC \textsection∗ 109 does not have a geographical orientation and that first sales abroad are protected by the doctrine. While this case was primarily focused on foreign purchases and subsequent importation,
        \footnote{Daniel Fisher. \emph{Supreme Court Upholds Right to Sell Foreign-Published Books.} Retrieved from \href{https://www.forbes.com/sites/danielfisher/2013/03/19/supreme-court-upholds-right-to-sell-foreign-published-books/#791960e52ef6}{https://www.forbes.com/sites/danielfisher/2013/03/19/supreme-court-upholds-right-to-sell-foreign-published-books/#791960e52ef6}
      }
    Justice Breyer, writing for the majority, further affirmed the first sale doctrine by stating that ``reliance on the `first sale' doctrine is also deeply embedded in the practices of booksellers, libraries, museums, and retailers, who have long relied on its protection.''
    \footnote{The Supreme Court of the United States. \emph{Kirtsaeng v. John Wiley \& Sons, Inc., 568 U.S. 519} (2013). Retrieved from \href{https://www.supremecourt.gov/opinions/12pdf/11-697_4g15.pd}{https://www.supremecourt.gov/opinions/12pdf/11-697_4g15.pd}.
    }

  \section*{Worldwide Exhaustion in the US}
    The Kirtsaeng case explicitly secured the rights of purchasers to trade in imported materials. This is a notion known as worldwide exhaustion. After an item has undergone a legitimate first sale and is imported into the United States, the rights of the copyright holder are eliminated.
         \footnote{Owner's Rights Initiative. \emph{Our Issues.} Retrieved from \href{http://www.ownersrightsinitiative.org/background/}{http://www.ownersrightsinitiative.org/background/}.
      }
  \subsection*{Nature of Exhaustion in the European Union}
    While the exhaustion principle applies to sales into the United States, the European Union does not recognize this doctrine. Attempts to sell a copyrighted or trademarked good \emph{into} the EU require explicit permission or license from the holder of the copyrights. By contrast, under a doctrine known as regional exhaustion, copyrighted or trademarked goods sold \emph{within} the EU between member countries do not require these permissions.

\section*{Is Permission Required to Sell Copyrighted Sports Team Material at Retail?}
  It is not necessary to obtain permission or a license to sell MLB/NFL/etc. goods at retail. This is true whether the seller operates a brick-and-mortar physical location, sells online, or both. It \emph{is} necessary, however, to purchase the goods from MLB/NFL/etc. authorized licensees or distributors.
     \footnote{Scott Silcox. \emph{Does a Retailer need an NFL (or MLB or NBA or NHL or NCAA) license to sell licensed sports products from my bricks and mortar store and/or over the internet?} Retrieved from \href{http://licensedsports.blogspot.com/2012/03/does-retailer-need-nfl-or-mlb-or-nba-or.html}{http://licensedsports.blogspot.com/2012/03/does-retailer-need-nfl-or-mlb-or-nba-or.html}
    }
\section*{Locating Authorized Sellers}
	A directory of 1500+ licensees authorized to sell material featuring the copyrighted or trademarked material of major sporting teams and organizations can be found at:\\
  \texttt{\href{http://www.licensedsports.net/}{http://www.licensedsports.net/}}

\section*{Is a License Needed to Manufacture and Sell Products with Copyrighted or Trademarked Material to Retailers, Wholesalers, Distributors, etc.?}
  While is not necessary to obtain a license, permission, or to pay royalties on the sale of goods purchased from a licensee, it is necessary to obtain permission and/or pay royalties if the individual or business intends to make or manufacture new goods featuring copyrighted or trademarked logos/material.
    \footnote{Scott Silcox. \emph{Does a Retailer need an NFL (or MLB or NBA or NHL or NCAA) license to sell licensed sports products from my bricks and mortar store and/or over the internet?} Retrieved from \href{http://licensedsports.blogspot.com/2012/03/does-retailer-need-nfl-or-mlb-or-nba-or.html}{http://licensedsports.blogspot.com/2012/03/does-retailer-need-nfl-or-mlb-or-nba-or.html}

    }
\section*{Fair Use}
  As mentioned above, copyrights grant creators certain rights relating to their work. They may exclusively reproduce their material, create derivatives of that material, and display the work publicly.
    \footnote{17 US Code \textsection∗ 106 - Exclusive rights in copyrighted works. Retrieved from \href{https://www.law.cornell.edu/uscode/text/17/106}{https://www.law.cornell.edu/uscode/text/17/106}.
    }
    \footnote{Brian Farkas. \emph{The `Fair Use' Rule: When Use of Copyrighted Material Is Acceptable.} Retrieved from \href{https://www.nolo.com/legal-encyclopedia/fair-use-rule-copyright-material-30100.html}{https://www.nolo.com/legal-encyclopedia/fair-use-rule-copyright-material-30100.html}. \textbf{Note:} Fair Use is not a ``rule,'' it is a defense that can be used in the event that a party is allegedly infringing copyrights.
    }
  These rights are not absolute, however. Important exceptions are included in the doctrine of fair use. It should be noted that ``fair use'' is a defense that can be used to avoid prosecution for copyright infringement---it is not a codified rule but rather a series of tests that courts use to determine whether or not a particular use case falls under the fair use doctrine.
    \footnote{Adria Sacarino. \emph{How NOT to Break the Law While Marketing.} Retrieved from: \href{https://blog.kissmetrics.com/break-the-law-marketing/}{https://blog.kissmetrics.com/break-the-law-marketing/}
    }

  In the hypothetical case being treated here, a not-for-profit organization intends to purchase MLB/NFL/etc. licensed goods from an authorized licensee and then use those goods as raw material to produce products \emph{de novo} that feature the copyrighted or trademarked material in some way. The legality of the re-manufacturing or repurposing of copyrighted material will depend largely on whether such activity can successfully be established as fair use.

  If the use can be determined to be fair, the copyrighted or trademarked materials may be used without the need for obtaining permission from the rights holder. Analysis of fair use cases is done based on four factors, each of which receive equal weight:
       \footnote{Adria Sacarino. \emph{How NOT to Break the Law While Marketing.} Retrieved from: \href{https://blog.kissmetrics.com/break-the-law-marketing/}{https://blog.kissmetrics.com/break-the-law-marketing/}.
       }
    \begin{itemize}
      \item The intended character and purpose of the use;
      \item the nature of the work being copied or derived from;
      \item the amount and substance of the use; and
      \item the effect that the use has on the original work's value.
    \end{itemize}
  The objective in determining fair use is to strike a balance between the rights holder and the rights of the public. Fair use is technologically agnostic and therefore the same test can be applied to any medium.

  It is critical that the test in Table \ref{tab:fair-use-test} be well-understood and applied carefully. In the event that a rights holder files suit claiming that their rights have been infringed upon, it is possible for a court to reduce or eliminate the amount of damages claimed if the alleged infringing party is able to demonstrate that they performed and understood their own analysis of the test and arrived at a good faith conclusion that their activity qualifies as fair.

  \begin{table}
  \centering
  \caption{Fair Use Test Matrix}
     \footnote{Purdue University Copyright Office. \emph{Copyright Exceptions: Fair Use.} Retrieved from \href{https://www.lib.purdue.edu/uco/CopyrightBasics/fair_use.html}{https://www.lib.purdue.edu/uco/CopyrightBasics/fair_use.html}
    }
  \vspace{2ex}
  \begin{tabularx}{\linewidth}{|X|c|c|c|c|} \hline
   & Purpose of Use & Nature of Work & Quantity Used & Market Effect\\ \hline\hline
   Fair Use & Nonprofit & Fact & Small Amount & No Effect \\
   & Educational & Published & & Licensing Not Avail. \\
   & Personal & & & Permission Not Avail. \\
   & Teaching & & & \\
   & Criticism & & & \\
   & Commentary & & & \\
   & Scholarship & & & \\
   & Research & & & \\
   & News Reporting & & & \\ \hline \hline
  Permission & Commercial & Creative & Large Amount & Major Effect \\
    & For Profit & Unpublished & Heart of the Work & Available to the world \\
    & Entertainment & & & \\ \hline
\end{tabularx}
  \label{tab:fair-use-test}
  \end{table}

 When applying the fair use test, the more factors that favor fair use for the intended use, the better off the case is.
    \begin{itemize}
        \item One factor (a single column in Table \ref{tab:fair-use-test}) in favor of fair use fails the test and would require permission;
        \item two factors is a tie and it is up the user to assess the risk;
        \item three factors would indicate that the intended use is more than likely fair use; and
        \item satisfying all four factors would constitute fair use.
    \end{itemize}

\section*{Copyfraud}
Copyfraud constitutes false or inappropriate allegations of copyright infringement. Works in the public domain, such as William Shakespeare's plays, Monet's paintings, etc., are frequently subject to false claims of infringement by the archives, libraries, and museums that own them. Some unknowing people and businesses, faced with a strongly-worded letter threatening litigation and steep fines, will end up obtaining licenses and/or paying fees to use or reproduce works that are, in fact, free for everyone to use.

The very nature of the copyright laws creates the incentive to perpetuate copyfraud. There are no civil penalties imposed on a party for falsely claiming ownership of material that is in the public domain. Users who do end up ceasing their activity or paying fees for permission also have no remedy under the current copyright laws. Falsely claiming copyright is technically a criminal act, but prosecutions of these types of cases is very rare.
    \footnote{Jason Mazzone. \emph{CopyFraud}. Retrieved from \href{http://www.nyulawreview.org/issues/volume-81-number-3/copyfraud}{http://www.nyulawreview.org/issues/volume-81-number-3/copyfraud}.
    }



\section*{Disclaimer; Limitation of Liability}
\emph{It should be noted that the author is not an attorney and this document was compiled as a result of research efforts. Nothing contained herein may be construed as legal advice or a legal opinion. The author cannot guarantee that this work is free from errors and therefore cannot be held liable for any damages, lost profits, or increased operating expenses resulting from any action taken, or not taken, on the basis of the contents of this memorandum.}
\begingroup
\nocite{*}
\raggedright
\bibliography{bib.bib}
\endgroup

\end{document}
