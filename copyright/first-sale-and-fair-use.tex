\documentclass[letterpaper,11pt]{texMemo}

\usepackage{graphicx}

\memoto{P. Ellison, Executive Director, Hand2Paw Foundation}
\memofrom{sb swae, Researcher}
\memosubject{First sale doctrine: re-sale of copyrighter material}
\memodate{06 May 2018}
\logo{\includegraphics[width=0.25\textwidth]{hand2paw.png}}

\begin{document}
\maketitle

Reference is to a hypothetical situation in which a 501(3)c non-for-profit coroporation seeks information as to whether or not it may procure MLB, NFL, or other major sporting organization-branded material and make, manufacture, distrubute, or sell derivtative works or goods using the licesned, copyrighted, or trademarked material as raw stock for its \emph{de novo} goods. This memorandum seeks to provide information, for educational purposes only, regarding the legal issues pertaing to such acitivities.

A detailed survey has not been completed at this time.  The information presented in this memorandum constitutes the product of research using publicly-available kjshkajshdkjsab been obtained so far from knowledgeable and responsible sources makes this summary reply possible.

The following conclusions can be reported:

\begin{itemize}
 \item Partially due to the US Supreme Court's decision in 2013 reaffirming the first sale doctrine, it is generally possible for a person to purchase copyrighted or trademarked material and resell it to other persons.
 \item Material that sees its first sale made pursuant to a license may be subject further restrictions on sale.
 \item Broadly speaking, it is not legal for a person to manufacture or otherwise make and subsequently sell derivative works using copyrighted or trademarked material without the permission of the rights holder. The copyright laws do contain exceptions to this rule.
 \item Procuring licensed, copyrighted, or trademarked raw material and making, manufacturing, distributing, or selling derivatitve works made from the source material will almost certainly have to rely on a "fair use" defense in order to assert legality.
\end{itemize}

As for specific questions surrounding this matter, the following brief answers develop from the studies made during the past few days. These conclusions are subject to expansion and more detailed examination should the survey be authorized to continue.

\section{Overview of Copyright}
\section{First Sale Doctrine}
\section{Is Licensure Required to Sell Copyrighted Sports Team Material at Retail?}
	NO, but must purhcase from an authorize retailer
		You do not need a license of any kind from the NFL to sell NFL products at retail. This holds whether you are a bricks and mortar retailer, or an e-tailer, or both. All you need is to find NFL licensees (or distributors – see below) willing to sell their products to you. This answer may prompt two other questions
\section{Locating Authorized Retailers}
	Directory of 1500+ authorized licensees is available at http://www.licensedsports.net/login.php
\section{Is a License Needed to Manufacture and Sell Products with Copyrighted or Trademarked Material to Retailers, Wholesalers, Distributors, etc.?}
  YES
\section{US Law: Wordlwide Exhaustion}
What is worldwide exhaustion?
	Copyrighted or trademarked goods legally purchased abroad and imported into the United States lose their protection upon entry into the US.
  \subsection{Nature of Exhaustion in the European Union}
  Does worldwide exhaustion apply to the European Union?
	NO. It is legal to resell copyrighted trademarked material \emph{within} the EU (regional exhasution), but not \emph{into} the EU
\section{Copyfraud}
  What is copyfraud?








\end{document}
