\documentclass[letterpaper,11pt]{texMemo}

\usepackage{graphicx}

\memoto{Penny Ellison, Executive Director, Hand2Paw Foundation}
\memofrom{sb swae, Researcher}
\memosubject{First sale doctrine: re-sale of copyrighter material}
\memodate{\today}
\logo{\includegraphics[width=0.25\textwidth]{hand2paw.png}}

\begin{document}
\maketitle

Reference is to a hypothetical situation in which a 501(3)c non-for-profit coroporation seeks information as to whether or not it may procure MLB, NFL, or other major sporting organization-branded material and make, manufacture, distrubute, or sell derivtative works or goods using the licesned, copyrighted, or trademarked material as raw stock for its \emph{de novo} goods. This memorandum seeks to provide information, for educational purposes only, regarding the legal issues pertaing to such acitivities.

A detailed survey has not been completed at this time.  The information presented in this memorandum constitutes the product of research using publicly-available kjshkajshdkjsab been obtained so far from knowledgeable and responsible sources makes this summary reply possible.

The following conclusions can be reported:

\begin{itemize}
 \item Partially due to the US Supreme Court's decision in 2013 reaffirming the first sale doctrine, it is generally possible for an individual or business entity to purchase copyrighted or trademarked material and resell it to other individuals or business entities.
 \item It is not legal for an individual or business entity to manufacture or otherwise make and subsequently sell goods using copyrighted or trademarked material without the permission of the holder of the rights.
 \item Material that is sold pursuant to a license may be subject further restrictions on sale.

\end{itemize}

As for specific questions surrounding this matter, the following brief answers develop from the studies made during the past few days.  These conclusions are subject to expansion and more detailed examination should it be decided that the survey should be continued.

What is Copyright?
What is the First Sale Doctrine?
Do I Need a License to Sell Copyrighted Material at Retail?
	NO, but must purhcase from an authorize retailer
		You do not need a license of any kind from the NFL to sell NFL products at retail. This holds whether you are a bricks and mortar retailer, or an e-tailer, or both. All you need is to find NFL licensees (or distributors – see below) willing to sell their products to you. This answer may prompt two other questions
Where can I find an authorized retailer for goods with sporting teams' copyrighted material?
	Directory of 1500+ authorized licensees is available at http://www.licensedsports.net/login.php
Do I Need a Licesne to manufacture and sell products with copyrighted or trademarked material to retailers, wholesalers, distributors, etc?
	YES
What is worldwide exhaustion?
	Copyrighted or trademarked goods legally purchased abroad and imported into the United States lose their protection upon entry into the US.
Does worldwide exhaustion apply to the European Union?
	NO. It is legal to resell copyrighted trademarked material \emph{within} the EU (regional exhasution), but not \emph{into} the EU
What is copyfraud?








\end{document}
