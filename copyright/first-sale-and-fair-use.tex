\documentclass[letterpaper,11pt]{texMemo}
\usepackage{graphicx}



\memoto{P. Ellison, Executive Director, Hand2Paw Foundation}
\memofrom{sb swae, Researcher}
\memosubject{First Sale Doctrine and the Fair Use Defense}
\memodate{06 May 2018}
\logo{\includegraphics[width=0.25\textwidth]{hand2paw.png}}

\begin{document}
\maketitle
\bibliographystyle{plain}
Reference is to a hypothetical situation in which a 501(3)c non-for-profit coroporation seeks information as to whether or not it may procure MLB, NFL, or other major sporting organization-branded material and make, manufacture, distrubute, or sell derivtative works or goods using the licesned, copyrighted, or trademarked material as raw stock for its goods. This memorandum seeks to provide information, for educational purposes only, regarding the legal issues pertaing to such acitivities.

A detailed survey has not been completed at this time.  The information presented in this memorandum constitutes the product of research using the publicly-available information obtained so far from knowledgeable and responsible sources makes this summary reply possible.

The following conclusions can be reported:

\begin{itemize}
 \item Partially due to a US Supreme Court decision in 2013 reaffirming the first sale doctrine, it is generally possible for a person to purchase copyrighted or trademarked material and resell it to other persons.
 \item Material that sees its first sale made pursuant to a license may be subject further restrictions on sale.
 \item Broadly speaking, it is not legal for a person to manufacture or otherwise make and subsequently sell derivative works using copyrighted or trademarked material without the permission of the rights holder. Exceptions do exist.
 \item Procuring licensed, copyrighted, or trademarked raw material and making, manufacturing, distributing, or selling derivatitve works made from the source material will almost certainly have to rely on a "fair use" defense in order to assert legality.
\end{itemize}

As for specific questions surrounding this matter, the following brief answers develop from the studies made during the past few days. These conclusions are subject to expansion and more detailed examination should the survey be authorized to continue.

\section*{Overview of Copyright}
Copyrights are a basket of rights that create a legal monopoly on the rights to sell, reproduce, and distribute a given work product. If one wishes to sell goods or material in which the copyright is held by another person or to sell a \emph{de novo} work product that incorporates or derives from such goods or material, two approaches are available. One may make use of exceptions in the copyright laws or one may obtain authorization from the holder of the copyrights \cite{carnes}.

\section*{First Sale Doctrine}
The first sale doctrine, which is codified at 17 USC § 109, is a statutory restriction that Congress has placed on the rights of copyright owners. It says that a person who purchases a piece or copy of copyrighted material or goods from the legal holder of the copyright then obtains the right to sell, display, or otherwise dispose of that particular copy \cite{17usc109}. Distribution rights end, however, once the owner (purchaser) sells that particular copy. The first sale doctrine does not offer protections for purchasers who make unauthorized reproductions of a copyrighted work.

The matter of first sale doctrine creates much confusion among copyright holders, purchasers, and prosecutors. In cases that do not involve reproduction of copyrighted materials, it is possible for purchasers/reseller to resist prosecution by making the claim that they reasonably believed that the goods they were selling had previously been subject to a legitimate first sale. Additionally, several criminal convictions related to copyrighted materials have been overturned owing to deficits in the prosecution's proof on the first sale.

The Owner's Right Initiative, an organization composed of individuals, businesses, and associations that seeks to protect ownership rights in the United States, distill the first sale doctrine down to their slogan, "You bought. You own it. You have a right to resell it." \cite{ownersrights}. It is on this basis that individuals may purchase MLB or NFL branded items and resell them on eBay.

  \subsection*{First Sale Doctrine in the Kirtsaeng v. John Wiley \& Sons Supreme Court Case}
  The United States Supreme Court decided a case in 2013 that reaffirmed the doctrine of first sale.

  A Thai graduate student was accused of infringing the copyrights of academic publisher John Wiley \& Sons because he made legitimate purcahses of Wiley textbooks in Thailand, where they were cheaper, imported them into the United States, and sold them online to US buyers. Wiley claimed that the sales were unauthorized. Kirtsaeng claimed that he had the right to sell the books because they were lawfully purchased in the first place.

  Lower courts ruled against Kirtsaeng on the basis that, because the books were manufactured in a foreign country, he had vioalted the copyright laws in the United States by conducting unauthorized sales.


\section*{Is Licensure Required to Sell Copyrighted Sports Team Material at Retail?}
	NO, but must purhcase from an authorize retailer
		You do not need a license of any kind from the NFL to sell NFL products at retail. This holds whether you are a bricks and mortar retailer, or an e-tailer, or both. All you need is to find NFL licensees (or distributors – see below) willing to sell their products to you. This answer may prompt two other questions
\section*{Locating Authorized Retailers}
	Directory of 1500+ authorized licensees is available at http://www.licensedsports.net/login.php
\section*{Is a License Needed to Manufacture and Sell Products with Copyrighted or Trademarked Material to Retailers, Wholesalers, Distributors, etc.?}
  YES
\section*{Worldwide Exhaustion in the US}
What is worldwide exhaustion?
	Copyrighted or trademarked goods legally purchased abroad and imported into the United States lose their protection upon entry into the US.
  \subsection*{Nature of Exhaustion in the European Union}
  Does worldwide exhaustion apply to the European Union?
	NO. It is legal to resell copyrighted trademarked material \emph{within} the EU (regional exhasution), but not \emph{into} the EU
\section*{Copyfraud}
  What is copyfraud?

\bibliography{bib.bib}

\end{document}
