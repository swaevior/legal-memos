\documentclass[letterpaper,11pt]{texMemo}

\usepackage{graphicx}

\memoto{Penny Ellison, Executive Director, Hand2Paw Foundation}
\memofrom{Sebastian B. Swae, Researcher}
\memosubject{First sale doctrine: re-sale of copyrighter material}
\memodate{\today}
\logo{\includegraphics[width=0.25\textwidth]{hand2paw.png}}

\begin{document}
\maketitle

\emph{It should be noted that the author is not an attorney and this document is a product of research. Nothing containe herein ought to be construed as legal advice or a legal opinion. The author cannot guarantee that this work is free from errors and therefore cannot be held liable for damages, lost profits, or increased operating expenses resulting from actions taken on the basis of the contents of this memorandum.}

Reference is to your May 4, 2018 memorandum, titled "Research Nonprofit's Ability to Sell Pet Merchandise Using NFL/MLB Logos," seeking information regarding the legal issues pertaining to\ purchase and resale of copyrighted material.

A detailed survey has not been completed at this time.  The examination will continue.  However, what has been obtained so far from knowledgeable and responsible sources makes this summary reply possible.

The following conclusions can be reported:

\begin{itemize}
 \item Partially due to the US Supreme Court's decision in 2013 reaffirming the first sale doctrine, it is generally possible for an individual or business entity to purchase copyrighted, or trademarked material and resell it to other individual or business entities.
 \item It is not legal for an individual or business entity to manufacture or otherwise make and subsequently sell goods using copyrighted or trademarked material without the permission of the holder of the rights.
 \item Material that is sold pursuant to a license may be subject further restrictions on sale.
 \item The U.S. can, if it will, firm up its objectives and employ its resources with a reasonable chance of attaining world leadership in space during this decade.  This will be difficult, but is possible, even recognizing the head start of the Soviets and the likelihood that they will continue to move forward with impressive successes.  In certain areas, such as communications, navigation, weather, and mapping, the US can and should exploit exploit its existing advanced position.
 \item If we do not make a strong effort now, the time will soon be reached when the margin of control over space and over men's minds through space accomplishments will have swung so far on the Russia side that we will not be able to catch up, let alone assume leadership.
 \item Even in those areas in which the Soviets already have the capability to be first and are likely to improve upon such capability, the United States should make aggressive efforts as the technological gains as well as the international rewards are essential steps in eventually gaining leadership.  The danger of long lags or outright omissions by this country is substantial in view of the possibility of great technological breakthroughs obtained from space exploration.
 \item Manned exploration of the moon, for example, is not only an achievement with great propaganda value, but it is is essential as an objective whether or not we are first in its accomplishment -- and we may be able to be first.  We cannot leapfrog such accomplishments, as they are essential sources of knowledge and experience for even greater successes in space.  We cannot expect the Russians to transfer the benefits of their experiences or the advantage of their capabilities to us.  We must do these things ourselves.
 \item The American public should be given the facts as to how we stand in the space race, told of our determination to lead that race, and be advised of the importance of such leadership to our future.
 \item More resources and more effort need to be put into our space program as soon as possible.  We should move forward with a bold program, while at the same time taking every practical precaution for the safety of the persons actively participating in space filghts.
\end{itemize}

As for specific questions surrounding this matter, the following brief answers develop from the studies made during the past few days.  These conclusions are subject to expansion and more detailed examination should it be decided that the survey should be continued.

What is Copyright?
What is the First Sale Doctrine?
Do I Need a License to Sell Copyrighted Material at Retail?
	NO, but must purhcase from an authorize retailer
		You do not need a license of any kind from the NFL to sell NFL products at retail. This holds whether you are a bricks and mortar retailer, or an e-tailer, or both. All you need is to find NFL licensees (or distributors – see below) willing to sell their products to you. This answer may prompt two other questions
Where can I find an authorized retailer for goods with sporting teams' copyrighted material?
	Directory of 1500+ authorized licensees is available at http://www.licensedsports.net/login.php
Do I Need a Licesne to manufacture and sell products with copyrighted or trademarked material to retailers, wholesalers, distributors, etc?
	YES
What is worldwide exhaustion?
	Copyrighted or trademarked goods legally purchased abroad and imported into the United States lose their protection upon entry into the US.
Does worldwide exhaustion apply to the European Union?
	NO. It is legal to resell copyrighted trademarked material \emph{within} the EU (regional exhasution), but not \emph{into} the EU
What is copyfraud?







\begin{itemize}
 \item Q1: Do we have a chance of beating the Sovietys by putting a laboratory in space, or by a trip around the moon, or by a rocket to land on the moon, or by a rocket to go to the moon and back with a man.  Is there any other space program which promises dramatic results in which we could win?
 \item A1: The Soviets now have a rocket capability for putting a multi-manned laboratory into space and have already crash-landed a rocket on the moon.  They also have the booster capability of making a soft landing on the moon with a payload of instruments, although we do not know how much preparation they have made for such a project.  As for a manned trip around the moon or a safe landing and return by a man to the moon, neither the US nor the USSR have such capability at this time, as far as we know.  The Russians have had more experience with large boosters and with flights of dogs and man.  Hence, they might be conceded a time advantage in circumnavigation of the moon and also in a manned trip to the moon.  However, with a strong effort, the United states could conceivably be first in those two accomplishments by 1966 or 1967.
 \item Q2: How much additional would it cost?
 \item A2: To start an acceleraged program with the aforementioned objectives clearly in mind, NASA has submitted an analysis indicating about 500 million would be needed for FY 1962 over and above the amount already requested of the Congress.  A program based upon NASA's analysis would, over a ten-year period, average approximately one billion a year above the current estimates of the existing NASA program.
 \item Q3: Are we working 24 hours a day on existing programs.  If not, why not?  If not, will you make recommendations to me as to how work can be speeded up?
 \item A3: There is not a 24-hour-a-day work schedule on existing NASA space programs except for selected areas in Project Mercury, the Saturn-C-1 booster, the Centaur engines and the final launching phases of most flight missions.  They advise that their schedules have been geared to the availabiltiy of facilities and financial resources, and that hence their over-time and 3-shift arrangements exist only in those activities in which there are particular bottlenecks or which are holding up operations in other parts of the programs.  For example, they have a 3-shift 7-day week operation in certain work at Cape Canaveral; the contractor for Project Mercury has averaged a 54-hour week and employs two or three shifts in some areas; Saturn C-1 at Huntsville is working around the clock during critical test periods while the remaining work on this project averages a 47-hour week; the Centaur hydrogen engine is on a 3-shift basis in some portions of the contractor's plants.  The work can be speeded up through firm decisions to go ahead faster if accompanied by additional funds needed for the acceleration.
 \item Q4: In building large boosters, should we put our emphasis on nuclear, chemical or liquid fuel, or a combination of these three?
 \item A4: It was the consensus that liquid, solid and nuclear boosters should all be accelerated.  The conclusion is based not only upon the necessity for back-up methods, but also because of the advantages of the different types of boosters for different missions.  A program of such emphasis would meet both so-called civilian needs and defense requirements.
 \item Q5: Are we making maximum effort?  Are we achieving the necessary results?
 \item A5: We are neither making maximum effort nor achieving results necessary if this country is to reach a position of leadership.
\end{itemize}

\includegraphics[width=0.4\textwidth]{Lyndon-Johnson-Signature.jpg}\\
\hspace*{0.9in} Lyndon B. Johnson


\end{document}
